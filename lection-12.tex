\documentclass[aspectratio=169]{beamer}
\usepackage[utf8]{inputenc}
\usepackage[english,russian]{babel}
\usepackage{cancel}
\usepackage{amssymb}
\usepackage{stmaryrd}
\usepackage{cmll}
\usepackage{graphicx}
\usepackage{amsthm}
\usepackage{amsmath}
\usepackage{tikz}
\usepackage{multicol}
\usetikzlibrary{patterns,calc}
\usepackage{chronosys}
\usepackage{proof}
\usepackage{multirow}
\usepackage{marvosym}
\usepackage{hyperref}
\setbeamertemplate{navigation symbols}{}
%\usetheme{Warsaw}

\newtheorem{thm}{Теорема}[section]
\newtheorem{dfn}{Определение}[section]
\newtheorem{lmm}{Лемма}[section]
\newtheorem{exm}{Пример}[section]
\newtheorem{snote}{Пояснение}[section]

\newcommand{\divisible}%
{\mathrel{\lower.2ex%
\vbox{\baselineskip=0.7ex\lineskiplimit=0pt%
\kern6pt \hbox{.}\hbox{.}\hbox{.}}%
}}

\begin{document}

\newcommand\doubleplus{+\kern-1.3ex+\kern0.8ex}
\newcommand\mdoubleplus{\ensuremath{\mathbin{+\mkern-10mu+}}}

\begin{frame}{}
\LARGE\begin{center}Алгебраические типы данных\end{center}
\end{frame}

\begin{frame}{Алгебра на типах данных}

\begin{center}\begin{tabular}{llll}
Множество & Мощность & Тип & Название\\\hline
$\varnothing$ & 0 & void & необитаемый\\
$\{\varnothing\}$ & 1 & unit & одноэлементный\\
$\{T,F\}$ & 2 & boolean & булевский, двухэлементный\\
$A \uplus B$ & $|\alpha|+|\beta|$ & Either Alpha Beta & тип-сумма\\
$A \times B$ & $|\alpha|\cdot|\beta|$ & (Alpha, Beta) & пара, декартово произведение\\
$B^A$ & $|\beta|^{|\alpha|}$ & Alpha $\rightarrow$ Beta & функциональный\\
\end{tabular}\end{center}

\begin{exm}
$\texttt{(boolean, A -> boolean)}$ соответствует $2\cdot(2^A)$
\end{exm}
\end{frame}

\begin{frame}{Алгебраический тип данных, тип-сумма}

\begin{dfn}
Отмеченным объединением множеств (дизъюнктным объединением) назовём:\vspace{-0.3cm}
$$A \uplus B  :=\{ \langle\ a, \text{``L'' } \rangle\ |\ a \in A \} \cup \{ \langle\ b, \text{``R'' }\rangle \ |\ b \in B\}
 \pause= \{ a_L\ |\ a \in A\} \cup \{ b_R\ |\ b \in B\}$$
\end{dfn}\vspace{-1cm}

\pause\begin{exm}\vspace{-0.5cm}
$$\mathbb{N} \cup \mathbb{N} = \{ 1, 2,3,\dots\}\quad\quad\mathbb{N} \uplus \mathbb{N} = \{ 1_L, 1_R, 2_L, 2_R, 3_L, 3_R, \dots \} $$
\vspace{-0.7cm}
$$\mathbb{N} \uplus \mathbb{Z} = \{ \dots -3_R, -2_R, -1_R, 0_R, 1_L, 1_R, 2_L, 2_R, 3_L, 3_R \dots \}$$
\end{exm}

Алгебраический тип данных (тип-сумма) задаётся набором конструкторов, каждому конструктору сопоставляется тип параметра.

\begin{exm}
\begin{tabular}{lll}
boolean := False | True & $B = \{\varnothing\} \uplus \{\varnothing\}$ & $\text{Л}: \varnothing_L$\\
angle := Degrees of int | Radians of real    & $A := \mathbb{Z} \uplus \mathbb{R}$ & $180^\circ: 180_L, \pi_R$
\end{tabular}
\end{exm}

\end{frame}

\begin{frame}[fragile]{Примеры из языков программирования}

\begin{multicols}{2}
\begin{verbatim}
type angle = record
     case radians : boolean of
         true: (rads: real);
         false: (degs: integer);
     end;

\end{verbatim}

\begin{verbatim}
struct angle {
     bool radians;
     union {
         float rads;
         int degs;
     }
};
\end{verbatim}
\end{multicols}
\vspace{-0.7cm}
Типичное применение:
\vspace{-0.2cm}\begin{verbatim}
union {
    short ax;
    struct {
        char al;
        char ah;
    }
};
\end{verbatim}
\end{frame}

\begin{frame}[fragile]{Списки}
\begin{itemize}
\item Список (целых чисел) --- алгебраический тип:
\begin{verbatim}
     type list = Nil | Cons of int * list
\end{verbatim}

\item Как строим значения:
\begin{verbatim}
     Nil                                =>  []
     Cons (5, Nil)                      =>  [5]
     Cons (3, Cons (4, Cons (5, Nil)))  =>  [3,4,5]
\end{verbatim}

%\item Множество list? 
%$$L = \{\varnothing\} \uplus (\mathbb{Z}\times L)$$
%Требуется найти неподвижную точку (это не совсем просто). 

\item Как используем значения:
\begin{verbatim}
     let rec length l = match l with
         Nil -> 0
       | Cons (_,lt) -> 1 + length lt
\end{verbatim}
\end{itemize}
\end{frame}

\begin{frame}[fragile]{Взглянем немного глубже}
Надо научиться строить и разбирать тип \verb!list = Nil | Cons of int * list!:

$$L = \{\varnothing\} \uplus (\mathbb{Z}\times L)$$

\begin{itemize}
\item Строить. Конструкторы: Nil, Cons --- или левая и правая инъекции ($In_L$, $In_R$).
$$Nil := In_L ()\quad\quad Cons\ a\ b := In_R\ \langle a, b \rangle$$

\item Разбирать.\\
\verb!let rec length l = match l with      !{\color{blue}\verb!match l with!}
\verb!    Nil -> 0                         !{\color{blue}\verb!|InL p -> 0!}
\verb!  | Cons (lh,lt) -> 1 + length lt    !{\color{blue}\verb!|InR p -> 1 + length (PrR p)!}\\\vspace{0.3cm}
В самом низу --- элиминатор Case:
$length\ l := Case\ l\ (\lambda p.0)\ (\lambda p.1 + length\ (\pi_R p))$
\end{itemize}
\end{frame}

\begin{frame}{Алгебраический тип как дизъюнкция}
Общие соображения: BHK-интерпретация. 
%Значение алгебраического типа построено, когда построен
%либо значение левого множества, либо правого множества, и мы знаем, какое. 

Интуиционистское исчисление высказываний
$$\infer{\Gamma\vdash\alpha\vee\beta}{\Gamma\vdash\alpha}\quad\quad\infer{\Gamma\vdash\alpha\vee\beta}{\Gamma\vdash\beta}\quad\quad\infer{\Gamma\vdash\gamma}{\Gamma\vdash\alpha\vee\beta\quad\Gamma\vdash\alpha\rightarrow\gamma\quad\Gamma\vdash\beta\rightarrow\gamma}$$

Просто-типизированное лямбда исчисление --- придумаем названия
$$\infer{\Gamma\vdash In_L A:\alpha\vee\beta}{\Gamma\vdash A:\alpha}\quad\quad\infer{\Gamma\vdash In_R B:\alpha\vee\beta}{\Gamma\vdash B:\beta}\quad\quad\infer{\Gamma\vdash \text{Case X L R}:\gamma}{\Gamma\vdash X:\alpha\vee\beta\quad\Gamma\vdash L:\alpha\rightarrow\gamma\quad\Gamma\vdash R:\beta\rightarrow\gamma}$$

\begin{exm}
Напомним, если $\tau = \varphi = \text{unit}$, то $\tau \vee \varphi \approx \text{bool}$.

Тогда $T^{\tau\vee\varphi} := In_L (),\quad F^{\tau\vee\varphi} := In_R ()$.
И, например, $\text{Not}\ x := \text{Case}\ x\ (\lambda t.In_R ())\ (\lambda t.In_L ())$
\end{exm}
\end{frame}

\begin{frame}{Реализация алгебраического типа}
Просто-типизированное лямбда исчисление:
$$\infer{\Gamma\vdash In_L A:\alpha\vee\beta}{\Gamma\vdash A:\alpha}\quad\quad\infer{\Gamma\vdash In_R B:\alpha\vee\beta}{\Gamma\vdash B:\beta}\quad\quad\infer{\Gamma\vdash \text{Case X L R}:\gamma}{\Gamma\vdash X:\alpha\vee\beta\quad\Gamma\vdash L:\alpha\rightarrow\gamma\quad\Gamma\vdash R:\beta\rightarrow\gamma}$$

Предлагаем такую реализацию:
$$In_L := \lambda x.\lambda t.\lambda f.t\ x, \quad\quad In_R := \lambda x.\lambda t.\lambda f.f\ x\quad\quad Case := \lambda x.\lambda l.\lambda r.x\ l\ r$$
\vspace{-0.7cm}
$$Case\ (In_L\ X^{\tau})\ L^{\tau\rightarrow\gamma}\ R \twoheadrightarrow_\beta (In_L\ X)\ L\ R = (\lambda t.\lambda f.t\ X)\ L\ R\twoheadrightarrow_\beta (L\ X)^{\gamma}$$

А где здесь дизъюнкция? Ожидаем, что $(In_L\ X^{\tau}): \tau\vee\varphi$. А что на деле?
$$X : \tau \vdash \lambda t^{\tau\rightarrow\gamma}.\lambda f^{\varphi\rightarrow\gamma}.t\ X : (\tau\rightarrow\gamma)\rightarrow(\varphi\rightarrow\gamma)\rightarrow\gamma$$

<<Если некоторое утверждение $\gamma$ истинно {\color{red}всегда}, когда оно следует из истинности $\tau$ и $\varphi$ --- то либо $\tau$, либо $\varphi$ истинно>>.
Рассуждение не совсем формально, потому что не хватает {\color{red}кванторов по утверждениям}, использующимся неявно:
$$\forall \gamma.(\tau\rightarrow\gamma)\rightarrow(\varphi\rightarrow\gamma)\rightarrow\gamma$$
\end{frame}

\begin{frame}{Примеры алгебраических типов}
Булевские значения:
$$T_1 := In_L () = \lambda t.\lambda f.t\ ()\quad\quad F_1 := In_R () = \lambda t.\lambda f.f\ ()\quad\quad If_1 := \lambda b.\lambda t.\lambda e.b\ (\lambda p.t)\ (\lambda p.e)$$

Ну или когда аргумент опущен за ненадобностью:
$$T := \lambda t.\lambda f.t\quad\quad F := \lambda t.\lambda f.f\quad\quad If := \lambda b.\lambda t.\lambda e.b\ t\ e$$

Списки:
$$Nil := In_L 0\quad\quad Cons\ p\ q := In_R\langle p, q \rangle$$

Тогда $[1,3,5]$ превращается в 
$Cons\ 1\ (Cons\ 3\ (Cons\ 5\ Nil))$.

Для простоты раскроем полностью $[1] = Cons\ 1\ Nil$:
$$\lambda t.\lambda f.f (\lambda p.p\ (\lambda f.\lambda x.f\ x)\ (\lambda t.\lambda f.t\ (\lambda f.\lambda x.x)))$$
\end{frame}

\begin{frame}
\begin{center}\begin{tabular}{lll}
Мощность & Тип & Высказывание\\\hline
0 & $\bot$ & необитаемый тип\\
1 & $(): \text{unit}$ & одноэлементный тип\\
$|\alpha|+|\beta|$ & $\text{Either}\ A^\alpha\ B^\beta : \alpha \vee \beta$ & тип-сумма, дизъюнкция\\
$|\alpha|\cdot|\beta|$ & $(A^\alpha, B^\beta) : \alpha \with \beta$ & тип-произведение, конъюнкция\\
$|\beta|^{|\alpha|}$ & $\lambda x^\alpha.B : \alpha \rightarrow \beta$ & функциональный, импликация
\end{tabular}\end{center}
\end{frame}


\begin{frame}{}
\LARGE\begin{center}Мощность множеств\end{center}
\end{frame}

\begin{frame}{Отношения}
\begin{dfn}$A \times B := \{\langle a,b \rangle\ |\ a \in A, b \in B\}$

Бинарное отношение --- $R \subseteq A \times B$

Функциональное бинарное отношение (функция) $R$ --- такое, что $\forall x.x\in A\rightarrow\exists !y.\langle x,y\rangle \in R$

$R$ --- инъективная функция, если $\forall x.\forall y.\langle x,t\rangle \in R\with \langle y,t\rangle \in R \rightarrow x=y$.

$R$ --- сюръективная функция, если $\forall y.y \in B\rightarrow\exists x.\langle x,y\rangle\in R$.\end{dfn}
\end{frame}

\begin{frame}{Равномощные множества}
\begin{dfn}Множество $A$ \emph{равномощно} $B$ $(|A|=|B|)$, если существует биекция
$f: A \rightarrow B$.

Множество $A$ имеет мощность, не превышающую мощности $B$ $(|A|\le|B|)$, если существует инъекция $f: A \rightarrow B$.
\end{dfn}
\end{frame}

\begin{frame}{Теорема Кантора-Бернштейна}
\begin{thm}Если $|A| \le |B|$ и $|B| \le |A|$, то $|A| = |B|$.\end{thm}
Заметим, $f: A \rightarrow B$, $g: B \rightarrow A$ --- инъекции, но не обязательно $g(f(x)) = x$.
\begin{proof}
%Пусть $f: A \rightarrow B$ и $g: B \rightarrow A$. Построим биекцию в явном виде. %\begin{enumerate}

Избавимся от множества $B$: пусть $A_0 = A$; $A_1 = g(B)$; $A_{k+2} = g(f(A_k))$.

\vspace{-0.2cm}
\begin{center}\tikz{
\node[inner sep=0, outer sep=0] (A0) at (0,0) {};
\node[inner sep=0, outer sep=0] (A1) at (2,0) {};
\node[inner sep=0, outer sep=0] (A2) at (3,0) {};
\node[inner sep=0, outer sep=0] (A3) at (3.5,0) {};
\node[inner sep=0, outer sep=0] (A4) at (3.75,0) {};
\node[inner sep=0, outer sep=0] (AN) at (4,0) {};
\node[inner sep=0, outer sep=0] (AE) at (6,0) {};

\node (B0) at (0,-1.5) {};
\node (B1) at (2,-1.5) {};
\node (B2) at (3,-1.5) {};
\node (B3) at (3.5,-1.5) {};
\node (B4) at (3.75,-1.5) {};
\node (BN) at (4,-1.5) {};
\node (BE) at (6,-1.5) {};

\fill[gray!80] ($(A0)+(0,0.1)$) rectangle node[pos=0.1,above]{\color{gray} $A_0$} ($(AE)+(0,0.15)$);
\fill[gray!30] ($(A1)+(0,0.2)$) rectangle node[pos=0.1,above]{\color{gray} $A_1$} ($(AE)+(0,0.25)$);
\fill[gray!80] ($(A2)+(0,0.3)$) rectangle node[pos=0.1,above]{\color{gray} $A_2$} ($(AE)+(0,0.35)$);
\fill[gray!30] ($(A3)+(0,0.4)$) rectangle node[pos=0.1,above]{\color{gray} $A_3$} ($(AE)+(0,0.45)$);
\fill[gray!80] ($(AN)+(0,0.5)$) rectangle node[midway,above]{\color{gray} $\cap A_k$} ($(AE)+(0,0.55)$);
%\draw (A3) -- node[midway,above]{$\dots$} (AN) (AE);
%\draw (B0) to (BE);

%\draw (AN) to (BN);

\fill[gray!80] ($(B0)-(0,0.1)$) rectangle node[pos=0.1,below]{\color{gray} $B_0$} ($(BE)-(0,0.15)$);
\fill[gray!80] ($(B1)-(0,0.1)$) rectangle node[pos=0.1,below]{\color{gray} $B_1$} ($(BE)-(0,0.15)$);
\fill[gray!80] ($(B2)-(0,0.1)$) rectangle node[pos=0.1,below]{\color{gray} $B_2$} ($(BE)-(0,0.15)$);
\fill[gray!80] ($(B3)-(0,0.1)$) rectangle node[pos=0.1,below]{\color{gray} $B_3$} ($(BE)-(0,0.15)$);
\fill[gray!80] ($(BN)-(0,0.1)$) rectangle node[midway,below]{\color{gray} $\cap B_k$} ($(BE)-(0,0.15)$);

\fill[gray!30] (4,0) -- (6,0) -- (6,-1.5) -- (4,-1.5);

\fill[gray!30] (3,-1.5) -- (3.5,-1.5) -- (3.75,0) -- (3.5,0);
\fill[gray!30] (0,-1.5) -- (2,-1.5) -- (3,0) -- (2,0);
\fill[gray!30] (3.75,-1.5) -- (3.875,-1.5) -- (3.875+0.0625,0) -- (3.875,0);
\fill[gray!80] (0,0) -- (2,-1.5) -- (3,-1.5) -- (2,0);
\fill[gray!80] (3,0) -- (3.5,-1.5) -- (3.75,-1.5) -- (3.5,0);
\fill[gray!80] (3.75,0) -- (3.875,-1.5) -- (3.875+0.0625,-1.5) -- (3.875,0);
%\draw[dashed,->] (B3) -- (A4);

}\end{center}

\vspace{-0.4cm}

Тогда, если существует $h: A_0 \rightarrow A_1$ --- биекция, то тогда $g^{-1}\circ h: A \rightarrow B$ --- 
требуемая биекция.

%\item Построим биекцию $h: A_0 \rightarrow A_1$\end{enumerate}
\end{proof}

\end{frame}

\begin{frame}{Построение биекции $h: A_0 \rightarrow A_1$}
Пусть $C_k = A_k \setminus A_{k+1}$. Тогда $g(f(C_k)) = g(f(A_k))\setminus g(f(A_{k+1})) = A_{k+2}\setminus A_{k+3} = C_{k+2}$.

\begin{center}\tikz{
\node[inner sep=0, outer sep=0] (A0) at (0,0) {};
\node[inner sep=0, outer sep=0] (A1) at (2,0) {};
\node[inner sep=0, outer sep=0] (A2) at (3,0) {};
\node[inner sep=0, outer sep=0] (A3) at (3.5,0) {};
\node[inner sep=0, outer sep=0] (A4) at (3.75,0) {};
\node[inner sep=0, outer sep=0] (AN) at (4,0) {};
\node[inner sep=0, outer sep=0] (AE) at (6,0) {};

\node (B0) at (0,-1.5) {};
\node (B1) at (2,-1.5) {};
\node (B2) at (3,-1.5) {};
\node (B3) at (3.5,-1.5) {};
\node (B4) at (3.75,-1.5) {};
\node (BN) at (4,-1.5) {};
\node (BE) at (6,-1.5) {};

\fill[gray!80] ($(A0)+(0,0.1)$) rectangle node[pos=0.1,above]{\color{gray} $C_0$} ($(AE)+(0,0.15)$);
\fill[gray!30] ($(A1)+(0,0.1)$) rectangle node[pos=0.1,above]{\color{gray} $C_1$} ($(AE)+(0,0.15)$);
\fill[gray!80] ($(A2)+(0,0.1)$) rectangle node[pos=0.1,above]{\color{gray} $C_2$} ($(AE)+(0,0.15)$);
\fill[gray!30] ($(A3)+(0,0.1)$) rectangle node[pos=0.1,above]{\color{gray} $C_3$} ($(AE)+(0,0.15)$);
\fill[gray!80] ($(AN)+(0,0.1)$) rectangle node[midway,above]{\color{gray} $\cap A_k$} ($(AE)+(0,0.15)$);
\fill[white] ($(A4)+(0,0.1)$) rectangle ($(AN)+(0,0.15)$);
%\draw (A3) -- node[midway,above]{$\dots$} (AN) (AE);
%\draw (B0) to (BE);

%\draw (AN) to (BN);

%\fill[gray!80] ($(B0)-(0,0.1)$) rectangle node[pos=0.1,below]{\color{gray} $B_0$} ($(BE)-(0,0.15)$);
%\fill[gray!80] ($(B1)-(0,0.1)$) rectangle node[pos=0.1,below]{\color{gray} $B_1$} ($(BE)-(0,0.15)$);
%\fill[gray!80] ($(B2)-(0,0.1)$) rectangle node[pos=0.1,below]{\color{gray} $B_2$} ($(BE)-(0,0.15)$);
%\fill[gray!80] ($(B3)-(0,0.1)$) rectangle node[pos=0.1,below]{\color{gray} $B_3$} ($(BE)-(0,0.15)$);
%\fill[gray!80] ($(BN)-(0,0.1)$) rectangle node[midway,below]{\color{gray} $\cap B_k$} ($(BE)-(0,0.15)$);

\fill[gray!30] (4,0) -- (6,0) -- (6,-1.5) -- (4,-1.5);

\fill[gray!30] (3,-1.5) -- (3.5,-1.5) -- (3.75,0) -- (3.5,0);
\fill[gray!30] (0,-1.5) -- (2,-1.5) -- (3,0) -- (2,0);
\fill[gray!30] (3.75,-1.5) -- (3.875,-1.5) -- (3.875+0.0625,0) -- (3.875,0);
\fill[gray!80] (0,0) -- (2,-1.5) -- (3,-1.5) -- (2,0);
\fill[gray!80] (3,0) -- (3.5,-1.5) -- (3.75,-1.5) -- (3.5,0);
\fill[gray!80] (3.75,0) -- (3.875,-1.5) -- (3.875+0.0625,-1.5) -- (3.875,0);
%\draw[dashed,->] (B3) -- (A4);

}\end{center}

Тогда определим $h(x)$ следующим образом:

\tikz{
\node (F) at (-3,-1) {$h(x) = \left\{\begin{array}{ll} x, & x \in C_{2k+1} \vee x \in \cap A_k\\
                g(f(x)), & x \in C_{2k}\end{array}\right.$};


\node[inner sep=0, outer sep=0] (A0) at (0,0) {};
\node[inner sep=0, outer sep=0] (A1) at (2,0) {};
\node[inner sep=0, outer sep=0] (A2) at (3,0) {};
\node[inner sep=0, outer sep=0] (A3) at (3.5,0) {};
\node[inner sep=0, outer sep=0] (A4) at (3.75,0) {};
\node[inner sep=0, outer sep=0] (AN) at (4,0) {};
\node[inner sep=0, outer sep=0] (AE) at (6,0) {};

\node (B0) at (0,-1.5) {};
\node (B1) at (2,-1.5) {};
\node (B2) at (3,-1.5) {};
\node (B3) at (3.5,-1.5) {};
\node (B4) at (3.75,-1.5) {};
\node (BN) at (4,-1.5) {};
\node (BE) at (6,-1.5) {};

\fill[gray!80] ($(A0)+(0,0.1)$) rectangle node[pos=0.1,above]{\color{gray} $C_0$} ($(AE)+(0,0.15)$);
\fill[gray!30] ($(A1)+(0,0.1)$) rectangle node[pos=0.1,above]{\color{gray} $C_1$} ($(AE)+(0,0.15)$);
\fill[gray!80] ($(A2)+(0,0.1)$) rectangle node[pos=0.08,above]{\color{gray} $C_2$} ($(AE)+(0,0.15)$);
\fill[gray!30] ($(A3)+(0,0.1)$) rectangle node[pos=0.08,above]{\color{gray} $C_3$} ($(AE)+(0,0.15)$);
\fill[gray!80] ($(AN)+(0,0.1)$) rectangle node[midway,above]{\color{gray} $\cap A_k$} ($(AE)+(0,0.15)$);
\fill[white] ($(A4)+(0,0.1)$) rectangle ($(AN)+(0,0.15)$);
%\draw (A3) -- node[midway,above]{$\dots$} (AN) (AE);
%\draw (B0) to (BE);

%\draw (AN) to (BN);

%\fill[gray!80] ($(B0)-(0,0.1)$) rectangle node[pos=0.1,below]{\color{gray} $C_0$} ($(BE)-(0,0.15)$);
\fill[gray!30] ($(B1)-(0,0.1)$) rectangle node[pos=0.1,below]{\color{gray} $C_1$} ($(BE)-(0,0.15)$);
\fill[gray!80] ($(B2)-(0,0.1)$) rectangle node[pos=0.08,below]{\color{gray} $C_2$} ($(BE)-(0,0.15)$);
\fill[gray!30] ($(B3)-(0,0.1)$) rectangle node[pos=0.08,below]{\color{gray} $C_3$} ($(BE)-(0,0.15)$);
\fill[gray!80] ($(BN)-(0,0.1)$) rectangle node[midway,below]{\color{gray} $\cap A_k$} ($(BE)-(0,0.15)$);
\fill[white] ($(B4)-(0,0.1)$) rectangle ($(BN)-(0,0.15)$);

\fill[gray!30] (4,0) -- (6,0) -- (6,-1.5) -- (4,-1.5);
\fill[gray!30] (2,-1.5) -- (3,-1.5) -- (3,0) -- (2,0);
%\fill[gray!30] (0,-1.5) -- (2,-1.5) -- (3.5,0) -- (3,0);
\fill[gray!30] (3.5,-1.5) -- (3.75,-1.5) -- (3.75,0) -- (3.5,0);
\fill[gray!80] (0,0) -- (3,-1.5) -- (3.5,-1.5) -- (2,0);
\fill[gray!80] (3,0) -- (3.75,-1.5) -- (3.875,-1.5) -- (3.5,0);
%\fill[gray!80] (3.75,0) -- (3.875,-1.5) -- (3.875+0.0625,-1.5) -- (3.875,0);
%\draw[dashed,->] (B3) -- (A4);

}

\end{frame}

\begin{frame}{Кардинальные числа}
\begin{dfn}Кардинальное число --- наименьший ординал, не равномощный никакому меньшему:
$$\forall x.x \in c \rightarrow |x| < |c|$$\end{dfn}
\begin{thm}Конечные ординалы --- кардинальные числа.\end{thm}
\begin{dfn}Мощность множества $(|S|)$ --- равномощное ему кардинальное число.\end{dfn}
\end{frame}

\begin{frame}{Диагональный метод}
\begin{lmm}$|\mathbb{R}| > |\mathbb{N}|$\end{lmm}
\begin{proof}Рассмотрим $a \in (0,1)$ и десятичную запись: $0.a_0a_1a_2\dots$.
Пусть существует биективная $f: \mathbb{N}\rightarrow (0,1)$.
По функции найдём значение $\sigma$, не являющееся образом никакого натурального числа.

\begin{center}\begin{tabular}{cc|ccccccl}
 $n$ &  $f(n)$ & $f(n)_0$ & $f(n)_1$ & $f(n)_2$ & $f(n)_3$ & $f(n)_4$ & $f(n)_5$ & $\dots$ \\\hline
 $n_0$ &  0.3  &  \color{red}3    & 0    &  0  &  0  &  0  &  0 & $\dots$ \\
 $n_1$ & $\pi/10$ &  3  & \color{red}1    &  4  &  1  &  5  &  9 & $\dots$ \\
 $n_2$ & $1/7$   & 1 &   4    &  \color{red}2  &  8  &  5  &  7 & $\dots$ \\\hline\pause
       & $\sigma$ & 8 &  6 &        7 &   \multicolumn{4}{l}{$\dots \sigma_k = (f(n_k)_k+5) \% 10$}
\end{tabular}\end{center}

%Заметим, что при любом $n \in \mathbb{N}$ выполнено $|\sigma_n - f(n)_n| = 5$.
\end{proof}
\end{frame}

\begin{frame}{Теорема Кантора}
\begin{thm}$|\mathcal{P}(S)| > |S|$\end{thm}
\begin{proof}Пусть $S = \{a,b,c,\dots\}$

\begin{center}\begin{tabular}{c|cccl}
$n$ & $a \in f(n)$ & $b \in f(n)$ & $c \in f(n)$ & $\dots$ \\\hline
$a$ & \color{red}И & Л & И \\
$b$ &       Л & \color{red}Л& И \\
$c$ &   И & И & \color{red}И\\\hline
   & Л & И & Л & $y \notin f(y)$
\end{tabular}\end{center}\pause

Пусть $f: S \rightarrow \mathcal{P}(S)$ --- биекция. Тогда 
$\sigma = \{ y\in S\ |\ y\notin f(y)\}$. Пусть $f(x) = \sigma$.
Но $x \in f(x)$ тогда и только тогда, когда $x \notin \sigma$, то есть $f(x) \ne \sigma$.
\end{proof}
\end{frame}

\begin{frame}{О буквах}
\small\url{https://en.wikipedia.org/wiki/Proto-Sinaitic_script}
\begin{center}\includegraphics[scale=0.6]{letters}\end{center}
\end{frame}

\begin{frame}{Иерархии $\aleph_n$ и $\beth_n$}
\begin{dfn}$\aleph_0 := |\omega|$; $\aleph_{k+1} := \min\{ a\ |\ a\text{ -- ординал},\aleph_k < |a|\}$\end{dfn}
\begin{dfn}$\beth_0 := |\omega|$; $\beth_{k+1} := |\mathcal{P}(\beth_k)|$\end{dfn}

Континуум-гипотеза (Г.Кантор, 1877): $\aleph_1 = \beth_1$ (не существует мощности, промежуточной 
между счётной и континуумом).

Обобщённая континуум-гипотеза: $\aleph_n = \beth_n$ при всех $n$.

\begin{dfn}Утверждение $\alpha$ противоречит аксиоматике: $\vdash\alpha$ ведёт к противоречию.

Утверждение $\alpha$ не зависит от аксиоматики: $\not\vdash\alpha$ и $\not\vdash\neg\alpha$.\end{dfn}\pause

\begin{thm}[О независимости континуум-гипотезы, Дж.Коэн, 1963] Утверждение $\aleph_1 = \beth_1$
не зависит от аксиоматики ZFC.\end{thm}
\end{frame}

\begin{frame}{Примеры мощностей множеств}
\begin{center}\begin{tabular}{l|l}Пример & мощность\\\hline
$\omega$ & $\aleph_0$\\
$\omega^2$, $\omega^\omega$ & $\aleph_0$\\
$\mathbb{R}$ & $\beth_1$\\
все непрерывные функции $\mathbb{R}\rightarrow\mathbb{R}$ & $\beth_1$\\
все функции $\mathbb{R}\rightarrow\mathbb{R}$ & $\beth_2$
\end{tabular}\end{center}

\end{frame}

%\newcommand{\divisible}%                                                     
%{\mathrel{\lower.2ex%
%\vbox{\baselineskip=0.7ex\lineskiplimit=0pt%
%\kern6pt \hbox{.}\hbox{.}\hbox{.}}%
%}}

%\begin{document}

%\newcommand\doubleplus{+\kern-1.3ex+\kern0.8ex}
%\newcommand\mdoubleplus{\ensuremath{\mathbin{+\mkern-10mu+}}}

%\begin{frame}{}
%\LARGE\begin{center}Теорема Лёвенгейма-Сколема\end{center}
%\end{frame}

\begin{frame}{Как пересчитать вещественные числа (неформально)?}
\begin{enumerate}
\item Номер вещественного числа --- первое упоминание в литературе, т.е. $\langle j, y, n, p, r, c \rangle$:\\
j --- гёделев номер названия научного журнала (книги);\\
y --- год издания;\\
n --- номер;\\
p --- страница;\\
r --- строка;\\
c --- позиция\pause
\item Попробуете предъявить число $x$, не имеющее номера? Это рассуждение сразу даст номер.\\
\end{enumerate}
\end{frame}

\begin{frame}{Мощность модели и аксиоматизации}
\begin{dfn} Пусть задана модель $\langle D, F_n, P_n \rangle$ для некоторой теории первого порядка. 
Её мощностью будем считать мощность $D$.
\end{dfn}\pause

\begin{dfn} Пусть задана формальная теория с аксиомами $\alpha_n$. Её мощность --- мощность множества $\{\alpha_n\}$.
\end{dfn}\pause

\begin{exm} Формальная арифметика, исчисление предикатов, исчисление высказываний --- счётно-аксиоматизируемые.
\end{exm}
\end{frame}

\begin{frame}{Элементарная подмодель}
\begin{dfn}$\mathcal{M}' = \langle D', F'_n, P'_n \rangle$ --- элементарная подмодель $\mathcal{M} = \langle D, F_n, P_n \rangle$, 
если: \pause
\begin{enumerate}
\item $D' \subseteq D$, \pause $F'_n$, $P'_n$ --- сужение $F_n$, $P_n$ (замкнутое на $D'$). \pause
\item $\mathcal{M}\models \varphi(x_1,\dots,x_n)$ тогда и только тогда, когда $\mathcal{M}'\models \varphi(x_1,\dots,x_n)$
при $x_i \in D'$. \pause
\end{enumerate}
\end{dfn}

\begin{exm}Когда сужение $M$ не является элементарной подмоделью? \pause

$\forall x.\exists y.x \ne y$. Истинно в $\mathbb{N}$. \pause Но пусть $D' = \{ 0 \}$.
\end{exm}
\end{frame}

\begin{frame}{Теорема Лёвенгейма-Сколема}
\begin{thm}Пусть $T$ --- множество всех формул теории первого порядка. 
Пусть теория имеет некоторую модель $\mathcal{M}$.
Тогда найдётся элементарная подмодель $\mathcal{M'}$, причём $|\mathcal{M'}| = \max(\aleph_0, |T|)$.
\end{thm}\pause

\begin{proof} (Схема доказательства)
\begin{enumerate} 
\item Построим $D_0$ --- множество всех значений, которые упомянуты в языке теории. \pause
\item Будем последовательно пополнять $D_i$: $D_0 \subseteq D_1 \subseteq D_2 \dots$, следя за мощностью.
$D' = \cup D_i$.
\item Покажем, что $\langle D', F_n, P_n\rangle$ --- требуемая подмодель.
\end{enumerate}
\end{proof}
\end{frame}

\begin{frame}{Начальный $D_0$}
Пусть $\{f^0_k\}$ --- все 0-местные функциональные символы теории. \pause
\begin{enumerate}
\item $D_0 = \{ \llbracket f^0_k \rrbracket \}$, если есть хотя бы один $f^0_k$. \pause
\item Если таких $f^0_k$ нет, возьмём какое-нибудь одно значение из $D$. \pause
\end{enumerate}\pause

Очевидно, $|D_0| \le |T|$.
\end{frame}

\begin{frame}{Пополнение $D$}
Фиксируем некоторый $D_k$. Напомним, $T$ --- множество всех формул теории. Рассмотрим $\varphi \in T$.\pause
\begin{enumerate}
\item $\varphi$ не имеет свободных переменных --- пропустим. \pause
\item $\varphi$ имеет хотя бы одну свободную переменную $y$. \pause
\begin{enumerate}
\item $\varphi (y, x_1, \dots, x_n)$ при $y,x_i \in D_k$ бывает истинным и ложным --- ничего не меняем \pause
\item $\varphi (y, x_1, \dots, x_n)$ при $y \in D$ и $x_i \in D_k$ либо всегда истинен, либо всегда ложен --- ничего не меняем \pause
\item $\varphi (y, x_1, \dots, x_n)$ при $y,x_i \in D_k$ тождественно истинен или ложен, но при 
$y' \in D \setminus D_k$ отличается --- добавим $y'$ к $D_{k+1}$. \pause
Вместе добавим всевозможные $\llbracket\theta(y')\rrbracket$.
\end{enumerate}
\end{enumerate}\pause

Всего добавили не больше $|T| \cdot |D_k|$. \pause $|\cup D_i| \le |T| \cdot |D_k| \cdot |\aleph_0| = \max (|T|, |\aleph_0|)$
\end{frame}

\begin{frame}{$\mathcal{M}'$ --- элементарная подмодель}
Индукцией по структуре формул $\tau \in T$ покажем, 
что все формулы можно вычислить, и что $\llbracket \varphi \rrbracket_\mathcal{M'} = \llbracket \varphi \rrbracket_\mathcal{M}$.\pause

\begin{enumerate}
\item База, 0 связок. $\tau \equiv P(f_1(x_1,\dots,x_n),\dots,f_n(x_1,\dots,x_n))$. \pause Если $x_i \in D'$, то значит,
добавлены на некоторых шагах (максимальный пусть $t$). Поэтому в $D_{t+1}$ можно вычислить формулу, и её значение сохранилось. \pause
\item Переход. Пусть формулы из $k$ связок сохраняют значения. Рассмотрим $\tau$ с $k+1$ связкой. \pause
\begin{enumerate}
\item $\tau \equiv \rho \star \sigma$ --- очевидно. \pause
\item $\tau\equiv\forall y.\varphi(y,x_1,\dots,x_n)$. \pause 
Каждый $x_i$ добавлен на каком-то шаге --- максимум $t$. \pause 
Если $\varphi(y,x_1,\dots,x_n)$ бывает истинен и ложен при $y_t, y_f \in D$, то $y_t, y_f \in D_{t+1}$ (по построению). \pause
Поэтому, если $\mathcal{M}\not\models\forall y.\varphi(y,x_1,\dots,x_n)$, то и 
$\mathcal{M'}\not\models\forall y.\varphi(y,x_1,\dots,x_n)$. \pause
Если же $\varphi(y,x_1,\dots,x_n)$ не меняется от $y$, то тем более
$\llbracket \varphi \rrbracket_\mathcal{M'} = \llbracket \varphi \rrbracket_\mathcal{M}$. \pause
\item $\tau\equiv\exists y.\varphi(y,x_1,\dots,x_n)$ --- аналогично.
\end{enumerate}
\end{enumerate}
\end{frame}

\begin{frame}{<<Парадокс>> Сколема}
\begin{enumerate}
\item Как известно, $|\mathbb{R}| = |\mathcal{P}(\mathbb{N})| > |\mathbb{N}| = \aleph_0$. \pause Однако, ZFC --- теория со счётным
количеством формул. \pause
Значит, существует счётная модель ZFC, то есть $|\mathbb{R}| = \aleph_0$. \pause В чём ошибка? \pause
\item У равенств разный смысл, первое --- в предметном языке, второе --- в метаязыке. 
\end{enumerate}
\end{frame}

\end{document}
